\section{Linear Function Approximation with (Weighted) Mean Squared Error}

In this scheme, we'll have \(p^\star(s)\), a stationary distribution over states that assigns a weight to each state. Note that we use it instead of uniform weighting, i.e., \(\frac{1}{|S|}\), as some states may be visited less frequently or not at all by the agent. Then, we define the weighted mean squared error (MSE) as:
\[
e_{\text{MS}}(\mathbf{w}) 
= \mathbb{E}_{s \sim p^\star(s)}\left[\left(v_\pi(s) - \hat{v}(s;\mathbf{w})\right)^2\right].
\]
In discrete state spaces, we can write it down as:
\[
e_{\text{MS}}(\mathbf{w})
\;=\; \sum_{s \in \mathcal{S}} p^\star(s)\,\bigl(v_\pi(s) - \hat{v}(s;\mathbf{w})\bigr)^2.
\]
\[
e_{\text{MS}}(\mathbf{w})
\;=\; \sum_{s \in \mathcal{S}} p^\star(s)\,\bigl(v_\pi(s) - \mathbf{w}^\top \mathbf{f}(s)\bigr)^2.
\]
In matrix form:
\[
e_{\text{MS}}(\mathbf{w}) 
\;=\; (\mathbf{v}_\pi - F\,\mathbf{w})^\top D_{p^\star} \,(\mathbf{v}_\pi - F\,\mathbf{w}),
\]
where \(D_{p^\star}\) is a diagonal matrix with \(D_{p^\star}(s,s) = p^\star(s)\). 

To get the optimal weights \(\mathbf{w}^*\) that minimize the weighted MSE, we can use the gradient descent algorithm. The update rule for the weights is:
\[
\mathbf{w} \leftarrow \mathbf{w} - \alpha \nabla_{\mathbf{w}} e_{\text{MS}}(\mathbf{w}),
\]
where \(\alpha\) is the learning rate and \(\nabla_{\mathbf{w}} e_{\text{MS}}(\mathbf{w})\) is the gradient with respect to \(\mathbf{w}\) of \(e_{\text{MS}}(\mathbf{w})\).

Starting with the gradient of the weighted MSE:
\[
\nabla_{\mathbf{w}} e_{\text{MS}}(\mathbf{w}) = \nabla_{\mathbf{w}} \mathbb{E}_{s \sim p^\star(s)}\left[\left(v_\pi(s) - \hat{v}(s;\mathbf{w})\right)^2\right].
\]

We move the gradient inside the expectation:
\[
\nabla_{\mathbf{w}} e_{\text{MS}}(\mathbf{w}) = \mathbb{E}_{s \sim p^\star(s)}\left[\nabla_{\mathbf{w}} \left(v_\pi(s) - \hat{v}(s;\mathbf{w})\right)^2\right].
\]

Then, we apply the chain rule:
\[
\nabla_{\mathbf{w}} e_{\text{MS}}(\mathbf{w}) = \mathbb{E}_{s \sim p^\star(s)}\left[2 \left(v_\pi(s) - \hat{v}(s;\mathbf{w})\right) \left(- \nabla_{\mathbf{w}} \hat{v}(s;\mathbf{w})\right)\right].
\]

Introducing \(-\frac{1}{2}\) on both sides to eliminate the 2:
\[
\nabla_{\mathbf{w}} e_{\text{MS}}(\mathbf{w}) = - \mathbb{E}_{s \sim p^\star(s)}\left[\left(v_\pi(s) - \hat{v}(s;\mathbf{w})\right) \nabla_{\mathbf{w}} \hat{v}(s;\mathbf{w})\right].
\]

Since \(\hat{v}(s;\mathbf{w}) = \mathbf{w}^\top \mathbf{f}(s)\), we have \(\nabla_{\mathbf{w}} \hat{v}(s;\mathbf{w}) = \mathbf{f}(s)\):
\[
\nabla_{\mathbf{w}} e_{\text{MS}}(\mathbf{w}) = - \mathbb{E}_{s \sim p^\star(s)}\left[\left(v_\pi(s) - \mathbf{w}^\top \mathbf{f}(s)\right) \mathbf{f}(s)\right].
\]

In RL setting, compared with Supervised Learning (SL):

1. There are no training data, so \(v(s)\) is unknown. Thus, we need to approximate this with Temporal Difference (TD), Monte Carlo (MC), etc.

2. Data isn't i.i.d., but Markovian, and collected by the agent itself.

To address (1), we approximate \(v(s)\) by:
\[
v(s) \approx \mathbb{E}\left[\underbrace{(r(S, A) - \hat{g} + \hat{v}(s'; \mathbf{w})}_{\text{approximation of } v(s) \text{ based on BEE}} - \hat{v}(s; \mathbf{w})) \mathbf{f}(s)\right],
\]
where \(v(s)\) is approximated based on the Bellman Error Equation (BEE). This involves the approximation of \(\hat{v}(s; \mathbf{w})\), hence it's bootstrapping (using an estimation of a value to estimate another one).

And because it's an expectation, it's sampling-friendly, i.e., using a sample of \((s, a, r', s')\) to get a sampling approximation:
\[
    \approx \left(r(s, a) - \hat{g} + \hat{v}(s'; \mathbf{w}) - \hat{v}(s; \mathbf{w})\right) \mathbf{f}(s).
\]